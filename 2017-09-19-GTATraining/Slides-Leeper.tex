\documentclass[14pt]{beamer} %Makes presentation
%\documentclass[14pt, handout]{beamer} %Makes Handouts

\usetheme{Singapore} %Gray with fade at top
\useoutertheme[subsection=false]{miniframes} %Supppress subsection in header
\useinnertheme{rectangles} %Itemize/Enumerate boxes
\usecolortheme{seagull} %Color theme
\usecolortheme{rose} %Inner color theme

\definecolor{light-gray}{gray}{0.75}
\definecolor{dark-gray}{gray}{0.55}
\setbeamercolor{item}{fg=light-gray}
\setbeamercolor{enumerate item}{fg=dark-gray}

\setbeamertemplate{navigation symbols}{}
%\setbeamertemplate{mini frames}[default]
%\setbeamercovered{dynamics}
\setbeamerfont*{title}{size=\Large,series=\bfseries}

%\setbeameroption{notes on second screen} %Dual-Screen Notes
%\setbeameroption{show only notes} %Notes Output

\setbeamertemplate{frametitle}{\vspace{.5em}\bfseries\insertframetitle}
\newcommand{\heading}[1]{\noindent \textbf{#1}\\ \vspace{1em}}

% small footnotes
\setbeamerfont{footnote}{size=\tiny}

\usepackage{bbding,color,multirow,times,ccaption,tabularx,graphicx,verbatim,booktabs,fixltx2e}
\usepackage{colortbl} %Table overlays
\usepackage[english]{babel}
\usepackage[latin1]{inputenc}
\usepackage[T1]{fontenc}
\usepackage{lmodern}
\usepackage{alltt}

\usepackage{tikz}
\usetikzlibrary{positioning}
\usetikzlibrary{trees}




\author[]{Thomas J. Leeper}
\institute[]{
  \inst{}%
  Department of Government\\London School of Economics and Political Science
}

\title{\large Teaching Political Science to\\ (LSE) Undergraduates:\\ Lessons for GTAs}

\date[]{19 September 2017}

\begin{document}

\frame{\titlepage}

\frame{
    \frametitle{What is teaching?}
    
    \vspace{1.5em}
    \begin{center}
    {\LARGE \onslide<2->{Teaching $\neq$ lecturing!}}
    \end{center}
    
}

\frame{
    \frametitle{Alignment}
    
	\begin{itemize}\itemsep1em
	\item<2-> All courses have ILOs
		\begin{itemize}
		\item ``Intended Learning Outcomes''
		\end{itemize}
	\item<3-> All course activities should align with those goals
	
	\item<4-> Know the ILOs for your course and design lessons that help students to achieve those outcomes
	\end{itemize}
}

% alignment -> everything we do in university is a learning activity that should add up to some higher-order goal, this is a piece of the puzzle that gets students to learn the things we want them to learn
	% class activities should help studies obtain learning outcomes, often by \textit{applying} material from readings and lectures
	% know the learning outcomes for your course; if they're not clear, ask your lecturer about them
	% make a lesson plan - what is the point of this lesson? what will you do in this lesson? how does the content of the lesson achieve learning objectives?
	% think about what to tell your students about how this lesson connects to learning goals -> may be useful to have them occassionally reflect on their own learning

\frame{
    \frametitle{Aligned Class Teaching}
    \begin{itemize}\itemsep1em
        \item<2-> Write a lesson plan for yourself based on the ILOs
        \item<3-> Think about whether students are learning \textit{about} something or learning \textit{how to do} something
        \item<4-> Explicitly connect activities to ILOs
        	\begin{itemize}
        	\item Telling students about that connection
        	\item Self-reflective activities
        	\end{itemize}
    \end{itemize}
}

\frame{
    \frametitle{Instrumental Motivations}
    \begin{itemize}\itemsep1em
        \item<2-> Teaching is a lot of work!!
        \item<3-> Personal pay-offs to be a GTA
        	\begin{itemize}
        	\item Improving your own teaching skills
        	\item Exposure to readings
        	\item Chance to observe all aspects of teaching and learning
        	\item Preparation for your own eventual teaching
        	\end{itemize}
    \end{itemize}
}

% think of being a GTA as a way to create a course that you could teach in the future - even if it's not exactly this course, it should get you to think about:
	% ILOs
	% readings
	% organization/flow of the course
	% class and lecture activities
	% exams

\frame{
    \frametitle{The First Class}
    \begin{enumerate}\itemsep1em
        \item Administration
        	\begin{itemize}
        	\item<2-> What is the purpose of class?
        	\item<2-> Provide a class syllabus
        	\end{itemize}
        \item Excitement/Engagement
        	\begin{itemize}
        	\item<3-> Showcase what class will be like through an activity
        	\end{itemize}
        \item Tone-setting
        	\begin{itemize}
        	\item<4-> Give a good first impression
        	\item<4-> What kind of GTA are you going to be this term?
        	\end{itemize}
    \end{enumerate}
}

% The first class
	% this should be partly administrative but partly substantive
	% you may want to make a one-page ``class syllabus'' (your name, contact information, office hours, rules about contacting you, defining your role/s)
	% have an activity in-mind that sets the tone for the course: students are expected to have done the reading, students are expected to participate, students should not use their laptops
	% make an impression of what kind of person you want your students to think you are: are you confident and serious? are you funny and approachable? are you your first name or Mr/Ms/whatever? etc. Practice your intro with a friend!
	
\frame{
    \frametitle{The Remaining 19 Classes}
    
    What should you do with the remainder of the year?
    
    \begin{itemize}\itemsep1em
        \item<2-> Ask your lecturer/convenor!
        \item<3-> Collaborate with other GTAs
        \item<4-> Reflect on your own practice
        \item<5-> Steal. Steal. Steal.
    \end{itemize}
}

% The rest of the classes
	% Ask your lecturer what you should be doing - it's their job to tell you what you should be doing. Ask them for feedback on lesson ideas.
	% If you're in a big course, collaborate with the other GTAs especially if they're more experienced
	% Prepare in advance to the extent possible, but set aside 1 or more hours between lecture and class to prep
	% Don't hesitate to observe other teachers. (Get their permission to do so.)
	% Think about objectives -> design activities -> reflect on your own practice (what worked? what didn't?)
	% Good artists steal! Don't reinvent things -> try to take lesson plans off-the-shelf
		% Think-pair-share is the safest fall back activity
		% Small discussion groups
		% Student presentations - these are almost universally terrible
		% Individual instructor feedback versus group feedback versus peer feedback
		% eventually students will want to talk about summatives and exams; plan on leaving room for this early and late in LT
		% Occasionally, you may need to lecture (very, very rarely) because students are confused, etc.
	% feedback class activities, learning, problems, questions to your lecturer

\frame{
    \frametitle{Some Things to Think About}
    \begin{itemize}\itemsep1em
        \item<1-> Lack of perceived authority
        \item<2-> Attendance
        \item<3-> Laptops/distractions
        \item<4-> Excessive requests for help/meetings
        \item<5-> ``is this on the exam''-ism
        \item<6-> Silence
        \item<7-> Teaching evaluations
    \end{itemize}
}


\frame{}

\end{document}
